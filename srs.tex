\documentclass[12pt]{report}
\usepackage{geometry}
\usepackage{graphicx}
\usepackage{titling}

\newcommand{\subtitle}[1]{
	\posttitle{
		\par\end{center}
		\begin{center}\large#1\end{center}
		\vskip0.5em}
}

\geometry{margin=1in}

\setcounter{tocdepth}{2}

\title{Requirements Document}
\subtitle{CS 383 - Team \texttt{\textasciicircum teamname}}
\author{
     Casey Blair - \textit{blai1919@vandals.uidaho.edu} \\
\and Mason Fabel - \textit{fabe0940@vandals.uidaho.edu} \\
\and Tessa Saul - \textit{saul7812@vandals.uidaho.edu} \\
\and Robin Rakowski - \textit{rako4161@vandals.uidaho.edu} \\
\and Ronald Rodriguez - \textit{rodr3849@vandals.uidaho.edu} \\
\and Tyler Wittreich - \textit{witt0241@vandals.uidaho.edu} \\
}
\date{\today}


\begin{document}

\maketitle

\tableofcontents
Document Typsetting and Editing: \textbf{Mason Fabel}
\clearpage

\chapter{Functional Requirements}

\begin{section}{Abdulmajeed Alotabibi}
\begin{subsection}{Game Startup}
The game will start with front screen with picture for the game and under
it three choices: A) New Game, B) Load Game, and C) Join Game. The player
will be able to choose any of the three choice by clicking on it.

\begin{enumerate}
\item New Game

New Game will be a fresh start for the player and it will start look like
it is the first time for the player play the game even if the player
already played the game and saved the game but want to play again from the
beginning, that will be possible. Moreover, the player will get a message
which is system reports domainname/IP, so other players can join him.

\item Load Game

When the player click on load game it will take the player to last place
that the player saved before quit the game. Moreover the system will
reports domainname/IP for the player as message so other player can join in
the adventurer.

\item Join Game

One of the choices the player will be able to chose is join game auto join,
the server will check any other player who does not have partner, if it
find it will match them.
\end{enumerate}
\end{subsection}
\end{section}

\begin{section}{Casey Blair}
\begin{subsection}{Drop-in Drop-Out Functionality}
When users start the game, they will be asked if they want to join an
existing game hosted by another user. New users can join and leave other
users’ games without the need for all users to start a new multiplayer game. This allows
for a more fluid gaming experience as new users can come and go as they
wish without interrupting other users’ progress. And with up to 27 users
joining one game, this will be necessary to have a fluid gameplay
experience that isn’t constantly being interrupted by new users wanting to
join.
\end{subsection}
\end{section}

\chapter{Nonfunctional Requirements}
\begin{section}{Tessa Saul}
\begin{subsection}{Beginning Levels}
The first 1 through n levels of the game will be office training levels. There will be a focus on social interactions with non-player-characters in order to raise the player's social standing. Gaining enough social standing will gain their aid to the player. This section will have to do more with writing story, which is why it is a non-functional requirement. 
\end{subsection} 
\end{section}


\begin{section}{Ronald Rodriguez}
\begin{subsection}{The Use of "Dungeon"}
As this game is going to be set in a Sci-fi world, it would make sense, on 
initial thought, to erase the idea of using the term "dungeon" to describe 
the environment one is in during a mission, and instead referring to it as 
a "site" or a "facility" or something of that nature. We believe, however, 
that it may be too early in the planning stages of the game to make this 
decision definitively. We may very well end up having some "themed" levels 
that take place inside of a computer that could most definitely be referred 
to as dungeon-esque. 
\end{subsection}
\end{section}

\begin{section}{Robin Rakowski}

\begin{subsection}{Setting: General}
The game will start at the Spaceport America, a center of modern commercial travel located in the Jornada del Muerto region in New Mexico. The game's levels will be based on the layout of the major buildings, starting in the administrative sections of the Spaceport and moving towards the space operations buildings as the game progresses. This requirement was deemed non-functional, as it does not concern the specific behavior or functions of the game system.
\end{subsection}

\begin{subsection}{Setting: Midgame}
The game will begin in training levels which resemble ordinary office space. The focus of these levels will be on social standing and verbal sparring within a bureaucratic setting. These levels allow the possibility of "befriending" NPCs who will semi-autonomously follow the player and aim them during later levels. The interface for these levels with be ASCII art.
\end{subsection}

\begin{subsection}{Setting: Endgame}
Later levels will be set in a "cyberspace"-like setting which the player will be kicked out of upon "death." The focus of these levels will be acquisition of knowledge and resources, and combat with viruses. The interface for these levels will be tile-based graphics, with real-time movement and turn-based combat.
\end{subsection}


\end{section}



\chapter{Rejected Requirements}

\begin{section}{Mason Fabel}
\begin{subsection}{Threading}
It was originally proposed that the system will view each player as a
thread, allowing those threads to send and receive messages from a
controller thread which manages the game state. This requirement has been
rejected for being too specific. While this is a good solution to the
general gameplay requirements, we believe it is better to keep this as a
suggestion rather than a requirement.
\end{subsection}

\begin{subsection}{3D Graphics}
It was mentioned that expanding this project to include 3D graphics might
be a future possibility. We believe that designing this project in such a
way to intentionally leave this path open is a distraction from the main
purpose of the game, and thus this should not be a requirement.
\end{subsection}
\end{section}

\end{document}
