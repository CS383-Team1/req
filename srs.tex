\documentclass[12pt]{report}
\usepackage{geometry}
\usepackage{graphicx}
\usepackage{titling}

\newcommand{\subtitle}[1]{
	\posttitle{
		\par\end{center}
		\begin{center}\large#1\end{center}
		\vskip0.5em}
}

\geometry{margin=1in}

\setcounter{tocdepth}{2}

\title{Requirements Document}
\subtitle{CS 383 - Team \texttt{\textasciicircum teamname}}
\author{
     Casey Blair - \textit{blai1919@vandals.uidaho.edu} \\
\and Mason Fabel - \textit{fabe0940@vandals.uidaho.edu} \\
\and Tessa Saul - \textit{saul7812@vandals.uidaho.edu} \\
% TODO: Add your own name/email here
}
\date{\today}


\begin{document}

\maketitle

\tableofcontents
Document Typsetting and Editing: \textbf{Mason Fabel}
\clearpage

\chapter{Functional Requirements}

\begin{section}{Abdulmajeed Alotabibi}
\begin{subsection}{Game Startup}
The game will start with front screen with picture for the game and under
it three choices: A) New Game, B) Load Game, and C) Join Game. The player
will be able to choose any of the three choice by clicking on it.

\begin{enumerate}
\item New Game

New Game will be a fresh start for the player and it will start look like
it is the first time for the player play the game even if the player
already played the game and saved the game but want to play again from the
beginning, that will be possible. Moreover, the player will get a message
which is system reports domainname/IP, so other players can join him.

\item Load Game

When the player click on load game it will take the player to last place
that the player saved before quit the game. Moreover the system will
reports domainname/IP for the player as message so other player can join in
the adventurer.

\item Join Game

One of the choices the player will be able to chose is join game auto join,
the server will check any other player who does not have partner, if it
find it will match them.
\end{enumerate}
\end{subsection}
\end{section}

\begin{section}{Casey Blair}
\begin{subsection}{Drop-in Drop-Out Functionality}
When users start the game, they will be asked if they want to join an
existing game hosted by another user. New users can join and leave other
users’ games without the need for all users to start a new multiplayer game. This allows
for a more fluid gaming experience as new users can come and go as they
wish without interrupting other users’ progress. And with up to 27 users
joining one game, this will be necessary to have a fluid gameplay
experience that isn’t constantly being interrupted by new users wanting to
join.
\end{subsection}
\end{section}

\begin{section}{Mason Fabel}
\begin{subsection}{Startup}
Upon being launched, the game will present the player with a simple menu
consisting of three items: ``New Game'', ``Load Game'', and ``Join Game''.
If one of the first two items is selected, the player's hostname and IP
address will be reported so that other players can potentially join the
game. A thread will then begin listening for joins at port 4500 by default,
or another port if the player has previously configured the game to run
over an alternative port.
\end{subsection}
\end{section}

\chapter{Nonfunctional Requirements}

\chapter{Rejected Requirements}

\begin{section}{Mason Fabel}
\begin{subsection}{Threading}
It was originally proposed that the system will view each player as a
thread, allowing those threads to send and receive messages from a
controller thread which manages the game state. This requirement has been
rejected for being too specific. While this is a good solution to the
general gameplay requirements, we believe it is better to keep this as a
suggestion rather than a requirement.
\end{subsection}

\begin{subsection}{3D Graphics}
It was mentioned that expanding this project to include 3D graphics might
be a future possibility. We believe that designing this project in such a
way to intentionally leave this path open is a distraction from the main
purpose of the game, and thus this should not be a requirement.
\end{subsection}
\end{section}

\end{document}
